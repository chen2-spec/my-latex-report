
\documentclass[UTF8]{ctexart}

\usepackage{float}
\usepackage{amsmath}
\usepackage{cases}
\usepackage{cite}
\usepackage{graphicx}
\usepackage[margin=1in]{geometry}
\geometry{a4paper}
\usepackage{fancyhdr}
\pagestyle{fancy}
\fancyhf{}


\title{{\Huge 系统基础开发工具{\large\linebreak\\}}{\Large 实验二\linebreak\linebreak}}
%please write your name, Student #, and Class # in Authors, student ID, and class # respectively
\author{\\姓\ 名:陈\ 佳\ 玲\\
学\ 号: 23100021002\\
专\ 业:23级环境工程\\
}
\date{\today}

\newpage

\begin{document}

\pagenumbering{arabic}

%\begin{document}

\fancyhead[C]{系统基础开发工具}
\fancyfoot[C]{\thepage}

\maketitle
\tableofcontents
\newpage

\section{实验目的}
 \begin{enumerate}
    \item 学习Shell工具和脚本:通过课堂学习和课下探索,熟悉shell基本指令,认识脚本以及一些基本语法知识,强化对于bash命令的熟练度,掌握在Linux环境下使用Shell进行基本操作和自动化任务的能力。
    
    \item 掌握编辑器Vim的使用:通过实际操作,学习Vim的基本操作模式、常用命令和高级功能,能够熟练使用Vim进行文本编辑和代码编写,提高在命令行环境下的编辑效率。
    
    \item 培养系统开发基础能力:通过Shell脚本编写和Vim编辑器使用的结合,培养在Unix/Linux系统环境下进行软件开发的基础能力,为后续课程和项目开发打下坚实基础。
\end{enumerate}


\section{实验内容}

\subsection{Shell工具和脚本}

\subsubsection{目录显示以及创建文件夹}
(1)显示目录:ls
\begin{figure}[H]
    \centering
    \includegraphics[width=0.95\textwidth]{picture/目录.png}
    \caption{目录显示}
\end{figure}

(2)创建文件夹mkdir
\begin{figure}[H]
    \centering
    \includegraphics[width=0.95\textwidth]{picture/mkdir.png}
    \caption{创建文件夹并验证}
\end{figure}

\subsubsection{打印输出内容文本}
使用echo指令输出文本内容或写入

\begin{figure}[H]
    \centering
    \includegraphics[width=0.95\textwidth]{picture/echo.png}
    \caption{echo打印hello}
\end{figure}

\subsubsection{环境变量PATH}
使用echo指令打印PATH环境变量,使用which echo可以看到echo所在路径

\begin{figure}[H]
    \centering
    \includegraphics[width=0.95\textwidth]{picture/PATH.png}
    \caption{查看PATH}
\end{figure}

\subsubsection{打印当前目录并更改}
\begin{enumerate}
    \item 打印当前目录:pwd
    \item 更改当前目录:cd
\end{enumerate}

\begin{figure}[H]
    \centering
    \includegraphics[width=0.95\textwidth]{picture/pwd.png}
    \caption{pwd和cd指令}
\end{figure}

\subsubsection{创建新文件并删除}
使用touch指令创建一个新的文件t111.c,使用ls指令确定已经创建之后,使用rm指令删除文件
\begin{figure}[H]
    \centering
    \includegraphics[width=0.95\textwidth]{picture/touch和rm.png}
    \caption{touch和rm指令}
\end{figure}

\subsubsection{查看用户手册}
使用man+所需查看指令或操作,可以获得用户指南。如使用man echo查看echo指令用户手册
\begin{figure}[H]
    \centering
    \includegraphics[width=0.95\textwidth]{picture/man echo.png}%保证图片占满页面宽度,且一致
    \caption{man echo}
\end{figure}

\begin{figure}[H]
    \centering
    \includegraphics[width=0.95\textwidth]{picture/echo用户手册.png}%保证图片占满页面宽度,且一致
    \caption{man echo指令结果}
\end{figure}

\subsubsection{查看和连接文件}
使用cat查看文件内容、连接多个文件内容到标准输出。如cat t1.txt t2.txt 查看 t1.txt 和 t2.txt 的内容。
\begin{figure}[H]
    \centering
    \includegraphics[width=0.95\textwidth]{picture/cat.png}%保证图片占满页面宽度,且一致
    \caption{cat指令}
\end{figure}

\subsubsection{创建脚本并添加内容}
使用指令touch 1.sh创建一个空的脚本文件,并使用vim编辑内容。在脚本中要把shell命令放到一个“脚本”当中,有一个要求:脚本的第一行必须写成类似这样的格式:

\verb|#!/bin/bash|
\begin{figure}[H]
    \centering
    \includegraphics[width=0.95\textwidth]{picture/操作.png}%保证图片占满页面宽度,且一致
    \caption{操作}
\end{figure}

\begin{figure}[H]
    \centering
    \includegraphics[width=0.95\textwidth]{picture/脚本文件内容.png}%保证图片占满页面宽度,且一致
    \caption{脚本文件内容}
\end{figure}

\subsubsection{运行脚本文件并使用chmod指令添加权限}
运行脚本文件1.sh缺少权限,使用ll指令查看权限之后,使用chnod指令添加权限,运行成功打印出hello world。
\begin{figure}[H]
    \centering
    \includegraphics[width=0.95\textwidth]{picture/chmod.png}%保证图片占满页面宽度,且一致
    \caption{脚本运行与chmod权限添加}
\end{figure}



\subsubsection{编写bash函数并使用}
编写两个 bash 函数 marco 和 polo 执行下面的操作。 每当你执行 marco 时,当前的工作目录应当以某种形式保存,当执行 polo 时,无论现在处在什么目录下,都应当 cd 回到当时执行 marco 的目录。 为了方便 debug,你可以把代码写在单独的文件 marco.sh 中,并通过 source marco.sh 命令,(重新)加载函数。通过 source 来加载函数,随后可以在 bash 中直接使用。

\begin{figure}[H]
    \centering
    \includegraphics[width=0.95\textwidth]{picture/marco.sh内容.png}
    \caption{marco.sh}
\end{figure}

\begin{figure}[H]
    \centering
    \includegraphics[width=0.95\textwidth]{picture/marco操作.png}
    \caption{使用操作}
\end{figure}

\subsubsection{压缩}
编写一个命令,它可以递归地查找文件夹中所有的 HTML 文件,并将它们压缩成 zip 文件。
\begin{figure}[H]
    \centering
    \includegraphics[width=0.95\textwidth]{picture/find压缩操作.png}
    \caption{操作}
\end{figure}

\begin{figure}[H]
    \centering
    \includegraphics[width=0.95\textwidth]{picture/创建html文件.png}
    \caption{创建的html文件}
\end{figure}

\begin{figure}[H]
    \centering
    \includegraphics[width=0.95\textwidth]{picture/压缩结果.png}
    \caption{压缩结果}
\end{figure}

\subsubsection{df查看磁盘空间使用情况}
df指令,显示文件系统的磁盘空间使用情况,监控磁盘空间。
\begin{figure}[H]
    \centering
    \includegraphics[width=0.95\textwidth]{picture/查看磁盘空间使用情况.png}
    \caption{查看磁盘空间使用情况}
\end{figure}

\subsubsection{进程}
(1)ps - 查看活动进程,显示当前系统中的活动进程,能够监控和管理进程。如,ps aux 显示系统中所有进程的详细列表。

\begin{figure}[H]
    \centering
    \includegraphics[width=0.95\textwidth]{picture/ps - 查看活动进程.png}
    \caption{ps - 查看活动进程}
\end{figure}

(2)top - 实时显示进程及系统资源的使用情况,动态监控系统和进程状态。如,直接运行 top 会打开一个交互界面,显示当前活动进程及资源使用情况。
\begin{figure}[H]
    \centering
    \includegraphics[width=0.95\textwidth]{picture/top - 实时显示进程动态.png}
    \caption{top - 实时显示进程动态}
\end{figure}


\subsection{编辑器Vim}


\subsubsection{三种模式}
基本上 vi/vim 共分为三种模式
\begin{enumerate}
    \item 命令模式:用户刚刚启动 vi/vim,便进入了命令模式。此状态下敲击键盘动作会被 Vim 识别为命令,而非输入字符,比如我们此时按下 i,并不会输入一个字符,i 被当作了一个命令。
    
    \item 输入模式:在命令模式下按下 i 就进入了输入模式,使用 Esc 键可以返回到普通模式。
    
    \item 命令行模式:在命令模式下按下 :(英文冒号)就进入了底线命令模式。底线命令模式可以输入单个或多个字符的命令,可用的命令非常多。按 ESC 键可随时退出底线命令模式。
    
\end{enumerate}

\subsubsection{创建my.txt文件}
输入命令vi+文件名便可创建文件并进入vi模式编辑器当中
\begin{figure}[H]
    \centering
    \includegraphics[width=0.95\textwidth]{picture/进入vi模式.png}
    \caption{vi编辑器页面}
\end{figure}

\subsubsection{编辑模式}
在一般模式之中,只要按下 i, o, a 等字符就可以进入输入模式了。
在编辑模式当中,你可以发现在左下角状态栏中会出现 –INSERT- 的字样,那就是可以输入任意字符的提示。
这个时候,键盘上除了 Esc 这个按键之外,其他的按键都可以视作为一般的输入按钮了,所以你可以进行任何的编辑。
\begin{figure}[H]
    \centering
    \includegraphics[width=0.95\textwidth]{picture/按下i并进入编辑模式.png}
    \caption{编辑模式}
\end{figure}|

\subsubsection{文档保存}
在一般模式中按下 :wq 储存后离开 vi
\begin{figure}[H]
    \centering
    \includegraphics[width=0.95\textwidth]{picture/退出并保存.png}
    \caption{退出并保存}
\end{figure}|


\subsubsection{打开指定文件并高亮显示关键词}
使用vim +/关键词+文件路径,打开指定文件并高亮显示关键词。如vim +/echo 1.sh
\begin{figure}[H]
    \centering
    \includegraphics[width=0.95\textwidth]{picture/查看高亮结果.png}
    \caption{查看高亮结果}
\end{figure}|

\subsubsection{调用外部命令}
切换到底行模式下,输入:!和外部命令,比如说外部命令ls,就是:!ls
\begin{figure}[H]
    \centering
    \includegraphics[width=0.95\textwidth]{picture/调用外部命令.png}
    \caption{调用外部命令}
    
\end{figure}|\begin{figure}[H]
    \centering
    \includegraphics[width=0.95\textwidth]{picture/外部命令调用结果.png}
    \caption{外部命令调用结果}
\end{figure}|


\subsubsection{替换}
切换到底行模式下,输入\verb|%s/string1/string2/g|, 把 string1 替换成string2, 下面把 hello替换为 world
\begin{figure}[H]
    \centering
    \includegraphics[width=0.95\textwidth]{picture/替换指令.png}
    \caption{替换指令}
    
\end{figure}|\begin{figure}[H]
    \centering
    \includegraphics[width=0.95\textwidth]{picture/替换结果.png}
    \caption{替换结果}
\end{figure}|
\section{心得体会}
通过本次实验,我实现了从理论到实践的跨越,深刻体会到Shell脚本与Vim编辑器在Linux环境下的核心。实验使我从零散命令的使用,到学会了用Shell脚本自动化复杂任务。同时,我从最初不适应Vim的模式切换,到熟练运用其命令完成快速编辑、查找和宏操作,深刻理解了Vim区分正常模式、插入模式和可视模式的设计哲学,切身感受到了其高效的编辑哲学。此次实验不仅是工具学习,更是一次思维训练,为我后续课程和项目开发提供了至关重要的底层技能与信心。


\section{实验代码查看链接}
本次报告相关练习、报告和代码均可以在https://github.com/chen2-spec/my-latex-report查看



\end{document}
