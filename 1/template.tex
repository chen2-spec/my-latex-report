
\documentclass[UTF8]{ctexart}

\usepackage{float}
\usepackage{amsmath}
\usepackage{cases}
\usepackage{cite}
\usepackage{graphicx}
\usepackage[margin=1in]{geometry}
\geometry{a4paper}
\usepackage{fancyhdr}
\pagestyle{fancy}
\fancyhf{}


\title{{\Huge 系统基础开发工具{\large\linebreak\\}}{\Large 实验一\linebreak\linebreak}}
%please write your name, Student #, and Class # in Authors, student ID, and class # respectively
\author{\\姓\ 名:陈\ 佳\ 玲\\
学\ 号: 23100021002\\
专\ 业:23级环境工程\\
}
\date{\today}
\maketitle
\newpage


%\title{基础物理实验报告}
%\author{\LaTeX\ by\ 驰雨Chiyuru}
%\date{\today}
\pagenumbering{arabic}

\begin{document}

%\fancyhead[L]{陈佳玲}
\fancyhead[C]{系统基础开发工具}
\fancyfoot[C]{\thepage}

\maketitle
\tableofcontents
\newpage

\section{实验目的}
 1. 学习版本控制(Git)
通过课堂学习和课下探索练习,能够掌握基本的Git指令,并且能够对git产生新的认识。
\\      
2. 学习 LaTeX 文档编辑
学习LaTeX 文档编辑器,练习相关的代码编写,掌握LaTeX的排版方法,能够学以致用,用LaTeX来生成自己的实验报告。


\section{实验内容}

\subsection{Git}

\subsubsection{创建本地库并初始化}
(1)创建本地库
\begin{figure}[H]
    \centering
    \includegraphics[width=0.95\textwidth]{picture/mkdir.png}
    \caption{创建仓库}
\end{figure}

(2)进入目录并初始化仓库
\begin{figure}[H]
    \centering
    \includegraphics[width=0.95\textwidth]{picture/init.png}
    \caption{初始化仓库}
\end{figure}

\subsubsection{创建新文件}
查看工作区和暂存区的状态,输出结果为状态报告。

\begin{figure}[H]
    \centering
    \includegraphics[width=0.95\textwidth]{picture/echo.png}
    \caption{创建新文件并添加内容}
\end{figure}

\subsubsection{检查当前文件的状态}
查看工作区和暂存区的状态,输出结果为状态报告。

\begin{figure}[H]
    \centering
    \includegraphics[width=0.95\textwidth]{picture/git status.png}
    \caption{查看文件状态}
\end{figure}

\subsubsection{将文件添加到暂存区}
(1)将工作区的文件更改添加到暂存区,准备下次提交
\begin{figure}[H]
    \centering
    \includegraphics[width=0.95\textwidth]{picture/git add ..png}
    \caption{添加文件到暂存区}
\end{figure}
(2)再次检查状态
\begin{figure}[H]
    \centering
    \includegraphics[width=0.95\textwidth]{picture/git status2.png}
    \caption{验证暂存区添加结果}
\end{figure}

\subsubsection{提交更新到本地仓库}
将暂存区的文件提交到版本历史中:
\begin{figure}[H]
    \centering
    \includegraphics[width=0.95\textwidth]{picture/git commit .png}%保证图片占满页面宽度,且一致
    \caption{提交更改}
\end{figure}

\subsubsection{查看提交历史}
验证提交已成功创建,显示提交历史,包括作者、日期和提交消息
\begin{figure}[H]
    \centering
    \includegraphics[width=0.95\textwidth]{picture/git log.png}%保证图片占满页面宽度,且一致
    \caption{验证提交}
\end{figure}

\subsubsection{创建分支}
使用指令“git branch feature-branch”创建分支,分支是指向特定提交的轻量级指针;创建分支不会复制文件,只是创建一个新指针;使用分支可以在不影响主线的情况下开发新功能。
\begin{figure}[H]
    \centering
    \includegraphics[width=0.95\textwidth]{picture/创建分支.png}%保证图片占满页面宽度,且一致
    \caption{创建分支}
\end{figure}

\subsubsection{切换分支}
使用指令“git switch feature-branch”切换分支,更新工作目录中的文件
\begin{figure}[H]
    \centering
    \includegraphics[width=0.95\textwidth]{picture/切换分支.png}%保证图片占满页面宽度,且一致
    \caption{切换分支}
\end{figure}

\subsubsection{合并分支}
使用git branch 指令查看分支之后,使用指令git merge feature-branch合并分支
\begin{figure}[H]
    \centering
    \includegraphics[width=0.95\textwidth]{picture/合并分支.png}%保证图片占满页面宽度,且一致
    \caption{合并分支}
\end{figure}

\subsubsection{删除文件}
使用git rm newrep.txt指令删除文件,并使用git status查看删除结果。
\begin{figure}[H]
    \centering
    \includegraphics[width=0.95\textwidth]{picture/删除文件.png}
    \caption{删除文件}
\end{figure}




\subsection{LaTeX}


\subsubsection{文档的结构}
(1)文档的类型定义:\verb|\documentclass{}| 
是LaTeX 文档的第一个命令。它控制了文档的页边距、字体大小等。文档的类型有article (文章)
report(报告), book(书籍), letter(信函), beamer(幻灯片)

(2)标记文档正文内容的开始和结束:
\verb|\begin{document} |和 \verb|\end{document}|

\subsubsection{文章的标题、作者、日期}
(1)标题\verb|\title{系统基础开发工具}|
\\(2)作者\verb|\author{陈佳玲}|
\\(3)日期\verb|\date{\today}|:能够生成当天的日期
\\(5)标题栏\verb|\maketitle|

\subsubsection{文章结构划分}
章节标题\verb|\section{}|,\verb| \subsection{}|,\verb|\subsubsection{}|

\subsubsection{文字样式}
(1)粗体\verb|\textbf{关键点}|
这是\textbf{关键点}
(2)斜体\verb|\textit{特别注意}|
需要\textit{特别注意}


\subsubsection{文本内部换行}
在段落内部进行手动换行使用\verb|\\|或\verb| \newline|


\subsubsection{数学公式}
在文本行中插入简洁的数学公式或符号,使用\verb|$...$|,例如\verb|$y=2x+1$|。若要凸显数学公式则使用\verb|\[ ... \]|,例如\verb|\[ x = \frac{-b \pm \sqrt{b^2 - 4ac}}{2a} \]|,\[ x = \frac{-b \pm \sqrt{b^2 - 4ac}}{2a} \]。,很多数学公式的实现还需要引入amsmath这个宏包,实现方法是在导言区添加\verb|\usepackage{amsmath}|

\subsubsection{创建有序表}
使用enumerate生成带自动编号的列表,用于需要表示顺序的步骤或条目。如\\ \verb|\begin{enumerate}|
   \\ \verb|\item 第一步|
  \\ \verb|\item 第二步| 
\\ \verb|\end{enumerate}||
生成结果\\
\begin{enumerate}
    \item 第一步
    \item 第二步
\end{enumerate}

\subsubsection{注释添加}
使用\verb|%|在代码中添加不会被编译和输出的注释,如输出\verb|这是一个普通句子。 % 这是关于上一句的注释|,结果为这是一个普通句子。 % 这是关于上一句的注释。

\subsubsection{特殊符号输入}
使用反斜杠\verb|\| 转义输出许多特殊字符,如\verb|#|等保留为命令符号。


\subsubsection{目录生成}
在LaTeX 中为了自动生成目录,我们需要在正文区添加\verb|\tableofcontents|,通常,为了让目
录和后面的正文内容不在同一页,我们可以加上\verb|\newpage|来实现另起一页

\section{心得体会}
通过学习Git,我掌握了版本控制的核心概念与基础操作。实践中的问题排查深化了我对工作流原理的理解,这项技能不仅提升了我的代码管理效率,更培养了我严谨协作的习惯。LaTeX的学习让我体验到“内容与格式分离”的专业排版。从文档结构构建到数学公式、表格的精准处理,我深刻感受到其自动化排版与交叉引用的强大优势。将它应用于实验报告撰写后,我彻底解放了格式调整的负担,能更专注于内容本身,使得文档格式更加美观标准。



\section{实验代码查看链接}
本次报告相关练习、报告和代码均可以在https://github.com/chen2-spec/my-latex-report.git查看



\end{document}
